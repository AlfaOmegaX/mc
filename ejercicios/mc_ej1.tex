
\documentclass[11pt]{article}

%%% PAQUETES

% Idioma
\usepackage[utf8]{inputenc}
\usepackage[spanish]{babel}
\setlength{\parindent}{0pt}

% Matemáticas
\usepackage{amssymb, amsthm}

% Fuentes
\usepackage[T1]{fontenc}
\usepackage[sfdefault, scaled=.85]{roboto}
\usepackage[scaled=.8]{FiraMono}
\usepackage[cmintegrals]{newtxsf}
\usepackage[italic]{mathastext}
\usepackage{textcomp}
\usepackage{wasysym}

% Tablas
\usepackage{multirow}
\usepackage{colortbl}

% Gráficos y colores

\usepackage[x11names, rgb, html]{xcolor}
\usepackage{graphics}
\usepackage{caption}
\usepackage{float}
\usepackage{adjustbox}

% Ajustes del documento
\usepackage{geometry}
\geometry{left=3cm,right=3cm,top=3cm,bottom=3cm,headheight=1cm,headsep=0.5cm}
\usepackage{enumitem}

% Entornos personalizados.
\usepackage{mdframed}

% Código
\usepackage{listingsutf8}

%%% COLORES

% Material Design

\definecolor{50}{HTML}{E0F7FA}
\definecolor{300}{HTML}{4DD0E1}
\definecolor{500}{HTML}{00BCD4}
\definecolor{700}{HTML}{0097A7}
\definecolor{900}{HTML}{006064}

% Solarized

\definecolor{sbase03}{HTML}{002B36}
\definecolor{sbase02}{HTML}{073642}
\definecolor{sbase01}{HTML}{586E75}
\definecolor{sbase00}{HTML}{657B83}
\definecolor{sbase0}{HTML}{839496}
\definecolor{sbase1}{HTML}{93A1A1}
\definecolor{sbase2}{HTML}{EEE8D5}
\definecolor{sbase3}{HTML}{FDF6E3}
\definecolor{syellow}{HTML}{B58900}
\definecolor{sorange}{HTML}{CB4B16}
\definecolor{sred}{HTML}{DC322F}
\definecolor{smagenta}{HTML}{D33682}
\definecolor{sviolet}{HTML}{6C71C4}
\definecolor{sblue}{HTML}{268BD2}
\definecolor{scyan}{HTML}{2AA198}
\definecolor{sgreen}{HTML}{859900}

% Colores del documento

\definecolor{text}{RGB}{78,78,78}
\definecolor{accent}{RGB}{129, 26, 24}

%%% LISTINGS

% Listing -> Código fuente
\renewcommand{\lstlistingname}{Código fuente}

% Ajustes para que funcionen bien las tildes de los comentarios

\lstset{
  inputencoding=utf8/latin1
}

% Ajustes de Listings para el documento

\lstset{
  frame=leftline,
  rulecolor=\color{300},
  framerule=2pt,
  % Números de línea
  numbers=left,
  % Margen adicional para alinear los entornos con el resto de párrafos
  xleftmargin=0.7em,
  % Espacio adicional debajo del título
  belowcaptionskip=1\baselineskip,
  % Colores
  basicstyle=\footnotesize\ttfamily\color{sbase00},
  keywordstyle=\color{700},
  commentstyle=\color{300},
  stringstyle=\color{500},
  numberstyle=\color{500},
  % Separar líenas largas en varias líneas
  breaklines=true,
  showstringspaces=false,
  tabsize=2,
}

%%% ENTORNOS PERSONALIZADOS

\newtheoremstyle{ejercicio-style} % Nombre del estilo.
{-0.5em}                          % Espacio por encima.
{}                                % Espacio por debajo.
{\normalfont}                     % Fuente del cuerpo.
{}                                % Identación.
{\bf\sffamily}                    % Fuente para la cabecera.
{.}                               % Puntuación tras la cabecera.
{.5em}                            % Espacio tras la cabecera.
{\thmname{#1}\thmnumber{ #2}\thmnote{ (#3)}}     % Especificación de la cabecera (actual: nombre en negrita).


\mdfdefinestyle{ejercicio-frame}{
  linewidth=2pt, %
  linecolor= 300, %
  backgroundcolor= 50,
  topline=false, %
  bottomline=false, %
  rightline=false,%
  leftmargin=0pt, %
  innerleftmargin=1em, %
  innerrightmargin=1em,
  rightmargin=0pt, %
  innertopmargin=1em,%
  innerbottommargin=1em, %
  splittopskip=\topskip, %
}%

\surroundwithmdframed[style=ejercicio-frame]{ejer}

\theoremstyle{ejercicio-style}
\newtheorem{ejer}{Ejercicio}

\begin{document}

% Cabecera del documento

\begin{tabular*}{\textwidth}{@{\extracolsep{\fill}}!{\color{300}{\vrule width 2pt}}>{\columncolor{50}}clc}
    \noalign{\global\arrayrulewidth=2pt}
    %\arrayrulecolor{300}\hline
    & & \\
    & \Large{Modelos de Computación (MC)} & \\
               & \large{Entrega de ejercicios.} & \\
               & \large{Tema 1} & \\
          & & \\
          & \textsf{Estudiante (nombre y apellidos): Miguel Lentisco Ballesteros}  & \\
          & \textsf{Grupo de prácticas: A1} & \\
          & \textsf{Fecha de entrega: 29/09/2017} & \\
         \multirow{-10}{*}{ \begin{tabular}{c}
        \small{3º curso / 1º cuatr.} \\ Grado Ing. Inform. \\ Doble Grado Ing. \\ Inform. y Mat.
\end{tabular}}  & & \\
    %\hline
\end{tabular*}

\vspace{1cm}

\section*{Ejercicios Tema 1}
\label{sec:ej_tema_1}

% Ejercicio 13
\begin{ejer}
   Dados dos homomorfismos $f:A^* \rightarrow B^*, g:A^* \rightarrow B^*$, se dicen que son iguales si $f(x)=g(x), \forall x \in A^{*}$. ¿Existe un procedimiento algorítmico para comprobar si dos homomorfismos son iguales?
\end{ejer}

\emph{Respuesta.} Tenemos que siendo $A = \{a_1,a_2,...,a_n\}$ un alfabeto, $A^* = \bigcup\limits_{i=0}^{\infty} A^i$, la cuestión es que por ser f y g homomorfismos cumplen que $f(uv)=f(u)f(v) \forall u,v \in A^*$ (igual para $g$), entonces solo tenemos que ver que $f(u)=g(u) \forall u \in A$. \\

Debido a que cualquier palabra en $A^*$ estará formada por la concatenación de los símbolos de $A^*$, y dado que podemos separarlas por ser las funciones homomorfas, si las dividimos en elementos de $A$ nos quedará que cualquier palabra de $A^*$ se forma con elementos de $A$, es decir; cualquiera palabra de $A^*$ se expresará como una secuencia de elementos de $A$.

Entonces, por ser homomorfas cumplen que $f(\epsilon) = g(\epsilon) = \epsilon$, luego el algoritmo que nos atañe es que comprueben todos los elementos del alfabeto, es decir, que cada símbolo se corresponda con el mismo para las dos funciones. \\

Es decir, por ser f y g homomorfas cumplen que $f(a_1...a_n)=f(a_1)...f(a_n)$ (válido para g también), entonces solo tenemos que ver para que f sea igual a g que se cumpla: $f(a)=g(a) \forall a \in A$ \\

En resumen, cualquier palabra de $A^*$ podemos separarla en una secuencia de elementos de $A$ y por las propiedades de ser homomorfas vistas arriba da igual el orden o el número de los elementos, la condición para que sean iguales es que tienen que llevar los mismos elementos de $A$ en $B$. \\

% Ejercicio 16
\begin{ejer}
	Dada la gramática $G = (\{S,A\},\{a,b\},P,S)$ donde $P = \{ S \rightarrow abAS,abA \rightarrow baab, S \rightarrow a, A \rightarrow b \}.$ Determinar el lenguaje que genera.
\end{ejer}

\emph{Respuesta.} Primero nos fijamos en como van a acabar las palabras generadas por este lenguaje, de $S$ podemos generar $abAS$ o $a$, en cualquier caso podemos seguir creando secuencias de $abA$ pero al final siempre habrá una $S$ última. Las $A$ se convierten en $b$ o bien $abA$ en $baab$, el caso es que el final de la palabra tendrá forzosamente que acabar en $a$. Lo siguiente es que lo que precede a esa a final puede ser cualquier patrón de $abb$ o $baab$ repetido e intercalándose. \\

Designamos $L_{a} = \{ (abb)^i : i\geq 0 \}$ y $L_{b} = \{ (baab)^i : i\geq 0 \} $ y ahora $L_{c} = \bigcup\limits_{i=0}^{\infty} {(L_a \cup L_b)}^i$, así tenemos cubiertas todas las posibilidades de concatenación de las dos subcadenas (incluída la palabra vacía) entonces: \\

El lenguaje generado por la gramática es $L(G) = \{ ua : u \in L_c \} \equiv$ todas las palabras que tienen una combinación cualquiera repetida e intercalada de las subcadenas $abb$ o $baab$ y que acaban en $a$. \\

$L(G) = \{ ua : u \in {\{abb,baab\}}^* \}$ \\

% Ejercicio 17
\begin{ejer}
	Sea la gramática $G=(V,T,P,S)$ donde:
	\begin{itemize}
		\item $V = \{ <numero>,<digito>\}$
		\item $T = \{0,1,2,3,4,5,6,7,8,9\}$
		\item $S = <numero>$
		\item Las reglas de producción $P$ son:
		\begin{itemize}
			\item $<numero>\rightarrow <numero><digito> $
			\item $<numero>\rightarrow <digito> $
			\item $<digito>\rightarrow 0|1|2|3|4|5|6|7|8|9 $
		\end{itemize}
	\end{itemize}
	Determinar el lenguaje que genera.
\end{ejer}

\emph{Respuesta.} Obviamente vemos que con $<numero>$ o convertimos a $<digito>$ o en $<numero><digito>$ el caso es que para obtener una palabra al final todos serán $<digito>$ que se convertirán en cifras de la base decimal. La conclusión a la que llegamos es que podemos generar todos los dígitos que queramos y además cualquier dígito puede ser cualquier cifra de la base decimal. Por tanto sacamos que las palabras generadas son cualquier numero natural en base decimal. \\

Lenguaje generado por la gramática es $L(G) = T^+ = \bigcup\limits_{i=1}^{\infty} T^i \equiv$ cualquier número natural en base decimal. \\

% Ejercicio 18
\begin{ejer}
	Sea la gramática $G = (\{A,S\},\{a,b\},S,P)$ donde las reglas de producción son:
	\begin{center}
		\begin{tabular}{cccc}
			$ S \rightarrow aS $ \\
			$ S \rightarrow aA $ \\
			$ A \rightarrow bA $ \\
			$ A \rightarrow b $
		\end{tabular}
	\end{center}
	Determinar el lenguaje generado por la gramática.
\end{ejer}

\emph{Respuesta.} Es fácil ver que si empezamos con $S$ podemos crear todas las $a$ que queramos con $S \rightarrow aS$ pero cuando hagamos $S \rightarrow aA$ será la última $a$, a partir de ahí solo podemos crear b, todas las que queramos, hasta que acabemos con $A \rightarrow b$ poniendo la última b. \\

Vemos claramente que podemos crear todas las $a$ que queramos (una mínimo) y luego todas las $b$ que queramos (una mínimo), siendo $ab$ la palabra de longitud mínima en el lenguaje generado ($\epsilon$ no está). \\

Luego el lenguaje generado por la gramática es: $L(G) = \{ a^i b^j : i,j \geq 1 \} \equiv$ cualquier palabra con una secuencia de $a$ seguida de una secuencia de $b$. \\

% Ejercicio 19
\begin{ejer}
	Encontrar si es posible una gramática lineal por la derecha o una gramática libre del contexto que genere el lenguaje $L$ supuesto que $L \subset {\{a,b,c\}}^{*}$ y verifica:
	\begin{itemize}
		\item $u \in L \iff u$ no contiene dos símbolos $b$ consecutivos.
		\item $u \in L \iff u$ contiene dos símbolos $b$ consecutivos.
		\item $u \in L \iff u$ contiene un número impar de símbolos $c$.
		\item $u \in L \iff u$ no contiene el mismo número de símbolos $b$ que de símbolos $c$
	\end{itemize}
\end{ejer}

\emph{Respuesta.} Todos los lenguajes generados son de tipo 3 (se pueden generar  con una gramática regular, pero que en concreto son también libre del contexto) menos el 4, aunque es del tipo 2 y tiene también una gramática libre del contexto.
	\begin{enumerate}
		\item Reglas:
			\begin{center}
				\begin{tabular}{cc}
					$ S \rightarrow aS | bX | cS | \epsilon $ \\
					$ X \rightarrow aS | cS | \epsilon $
				\end{tabular}
			\end{center}
		\item Reglas:
			\begin{center}
				\begin{tabular}{cc}
					$ S \rightarrow aS | bS | cS | bbX $ \\
					$ X \rightarrow aX | bX | cX | \epsilon $
				\end{tabular}
			\end{center}
		\item Reglas:
			\begin{center}
				\begin{tabular}{cc}
					$ S \rightarrow aS | bS | cX $ \\
					$ X \rightarrow aX | bX | cS | \epsilon $
				\end{tabular}
			\end{center}
		\item Reglas:
			\begin{center}
				\begin{tabular}{cccccc}
					$ S \rightarrow S_1 | S_2 $ \\
					$ S_1 \rightarrow B | BS_1 $ \\
					$ S_2 \rightarrow C | CS_2 $ \\
					$ B \rightarrow bX | cBB | aB $ \\
					$ C \rightarrow cX | bCC | aC $ \\
					$ X \rightarrow bC | aX | cB | \epsilon $
				\end{tabular}
			\end{center}
	\end{enumerate}

\end{document}